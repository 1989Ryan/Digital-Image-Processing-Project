\documentclass[UTF8]{ctexart}
\usepackage{geometry}
\usepackage{mathrsfs}
\usepackage{graphicx}
\usepackage{multirow}
\usepackage{amsmath}
\usepackage{amssymb}
\usepackage{array}
\usepackage{subfigure}
\usepackage{algorithm2e}
\usepackage{setspace}
\usepackage{diagbox}
\geometry{a4paper,scale=0.75}
\newcommand{\xiaosi}{\fontsize{12.1pt}{\baselineskip}\selectfont}
\newcommand{\sihao}{\fontsize{14.1pt}{\baselineskip}\selectfont}
\newcommand{\upcite}[1]{\textsuperscript{\textsuperscript{\cite{#1}}}}

%opening
\title{社会网络中的信息传播优化问题}
\author{柴嘉骏 \, 王适未 \, 杨岳东 \, 于铭瑞 \, 赵子瑞}
\begin{document}

\maketitle

\newenvironment{cnabandkey}[2][\sihao 摘要] % 定义中文摘要和关键词环境
{\newcommand{\ckeywords}{#2} %
	\begin{center} \bfseries #1 \end{center} %
	\begin{quotation}
	}{\paragraph{\sihao 关键词:} \textrm{\ckeywords} %
	\end{quotation}
}


\newenvironment{enabandkey}[2][\sihao Abstract] % 定义英文摘要和关键词环境
{\newcommand{\ekeywords}{#2} %
	\begin{center} \bfseries #1 \end{center} %
	\begin{quotation}
	}{\paragraph{\sihao Keywords:} \textrm{\ekeywords} %
	\end{quotation}
}

	\begin{cnabandkey}{社会网络信息传播\, 优化\, 概率\, 独立级联模型\, 爬山贪心\, CELF}
		\xiaosi
		社会网络中影响最大化问题是对于给定$k$值,寻找$k$个具有最大影响范围的节点集合,这是一个优化问题,并且是NP完全问题。该问题已经得到了广泛的关注和研究,但其中仍存在一些问题:现有的模型中很少有涉及传播时间的概念,但现实的信息扩散都是需要时间的;传播概率作为随机因素加到传播模型中,可能使得传播模型结果不稳定;在优化过程中也存在算法时间复杂度高、容易局部最优等问题。
		
		本文将社会网络中的信息传播优化问题分为两个部分,即社会网络信息传播模型和传播效果优化模型。对于社会网络传播模型,基于独立级联模型,我们提出了基于概率的社会网络传播模型。该模型充分考虑了传播时间和传播概率对传播效果的影响,并对其进行了准确地描述。通过计算每个节点被激活的概率,将传播概率从随机因素变成确定性因素,使传播模型结果稳定。对于传播效果优化模型,我们应用CELF算法优化的爬山贪心算法,并以通过传播模型得到的最终被激活节点数目的期望的边际值作为影响因子,使爬山贪心算法的结果更加准确,更接近全局最优。同时CELF算法降低了爬山贪心算法的时间复杂度,使算法适用于本问题规模。
		
		对于本问题中的问题一,应用本文中的模型和算法,选择10个最优初始激活节点,最终被激活节点数目的期望可以达到80.97个。对于问题二,应用本文中的模型和算法,最少选择557个初始激活节点,可以使得最终被激活节点数目的期望达到800个以上。
		
		
		
	\end{cnabandkey}
	
	\clearpage
	
\begin{spacing}{1.2}
\xiaosi\tableofcontents\newpage
\end{spacing}


\xiaosi
\section{引言}

社会网络是指由个体及个体之间的关系所组成的一个复杂网络,这种复杂的社会结构对信息的传播和扩散起着至关重要的作用。当一个人采纳一个新的思想或接受一种产品时,他会向他的朋友或同事推荐,某些人可能会接受或采纳他的推荐,并进一步向他们自己的朋友或同事推荐,这个过程称为传播或扩散(Propagation or Spreading)。一个人的行为在很大程度上取决于周边的朋友或同事的决定。

社会网络的传播和扩散过程在社会科学中已有很长的研究历史。Richardson和Domingos等人\upcite{Richardson}将影响最大化问题归纳为一个算法问题,即如何定位网络中某些最有影响力的成员,提供给他们免费的样品,希望通过他们向网络中其他成员推荐,从而达到营销的目的, 那么该如何选择这$k$个初始成员使得最终购买人数最多。影响最大化问题的研究有着十分重要的现实意义,在市场营销、广告发布、舆情预警以及社会安定等方面有十分重要的应用。

社会网络中影响最大化问题(即如何选择$k$个种子节点,使其在传播过程结束之后, 传播的范围达到最大)已被证明是一个NP-hard问题\upcite{Kempe}。对于该问题,可以使用一些常见的启发式节点选择策略,包括基于点的度数或中心度等。然而,就传播范围而言,完全基于度数的启发式规则的效果并不理想,因为该方法显然没有考虑到社会网络的传播特性。Kemple和Kleinberg提出了一种自然的爬山贪心算法,它在每一步都选择当前“最具影响力”的节点作为初始传播对象进行传播。所谓“最具影响力”的节点, 即是当前能够激活最多节点的节点。然而,选择“最具影响力”的节点是一个非常耗时的过程,对于大型社会网络,这种贪心算法由于高耗时并不适用。另外,如何定义节点的“影响力”也是一个值得探究的问题,对于同一网络,不同的传播时间和传播方式会导致同一节点的“影响力”发生变化。

本文第2节介绍社会网络和社会网络信息传播的基本知识,并介绍最常用的传播模型之一\raisebox{0.5mm}{------}独立级联模型;第4节介绍本文针对本项目具体问题提出的更加精准的社会网络信息传播模型;第5节介绍本文使用的传播影响优化模型;第6节介绍应用本文提出的模型和算法在本项目问题上的实验结果及结果分析;最后一节对本文和本问题进行总结,并对未来进行展望。



\section{背景知识}
\subsection{社会网络及信息传播}
社会网络是指社会个体成员之间因为互动而形成的相对稳定的关系体系。社会网络以个人为节点(node)构成社会结构,人与人之间通过相互依赖关系联结起来。相互依赖关系可能是朋友关系、同学关系、生意伙伴关系、种族信仰关系等。

一个社会网络可以用一张网络图来表示,其中节点(node)代表人,边(edge)代表人与人之间的关系。如果两节点之间的关系是双方对等的(例如朋友关系、同学关系等),则边为无向边;如果两节点之间的关系是不对等的(例如微博的关注关系、论文的引用关系等),则边为有向边,从一个节点指向另一个节点。

无向图和有向图如图所示 \ref{graphy} .

\begin{figure}[h!]
	\centering
	\subfigure[无向图]{
		\label{无向图} %% 第一幅图的标签
		\includegraphics[width=0.25\textwidth]{secondgraph.png}}
	\hspace{0.1in} \subfigure[有向图]{
		\label{训练损失函数} %% 第二幅图的标签
		\includegraphics[width=0.35\textwidth]{firstgraph.png}}
	\caption{无向图和有向图}
	\label{graphy} %% label for entire figure
\end{figure}

以下以微博网络为例,讨论信息在社会网络中的传播问题。微博网络结构示例见图 \ref{weibo}。

\begin{figure}[h!]
	\centering
	\includegraphics[width=0.3\linewidth]{weibo.png}
	\caption{微博网络结构示意图}
	\label{weibo}
\end{figure}

信息在社会网络中以个人节点为载体,沿着节点之间的边进行传播。信息传播的方向与边的指向一致。在网络中,从不同节点开始传播的信息,其传播效果可能大不相同。社会网络中的信息传播优化问题所要讨论的就是如何选择起始的传播节点,使得信息能获得最大范围的传播(或达到指定的范围)。

\subsection{独立级联模型}
社会网络中的信息传播过程可以用独立级联模型来描述。该模型将整个社会网络看做一个有向图$G(V,E)$,其中$G$是所有节点的集合,$E$是所有边的集合。每条边有一个传播概率$p (0 \leqslant p \leqslant 1)$,即信息有概率$p$可以沿着某条边从一个节点传播到另一个节点。在此假设所有边的传播概率都相同。

信息传播的过程如下:

在$t=0$时刻,信息从某些节点开始第一次传播。这些初始节点被认为是处于激活状态,构成初始激活集合$S_0$。

在$t=1$时刻,集合$S_0$中的节点$v\in S_{0}$可以将信息以概率$p$传播给它们未被激活的邻居节点$u\notin S_0$邻居节点即有边与之相连的节点,信息传播方向与边的指向一致)。如果传播成功,即$u$被激活,则$u$将被加入$t=1$时刻的激活集合$S_1$。集合$S_1$包含$S_0$中的所有节点,以及在$t_1$时刻被激活的所有节点。

\ldots \ldots

在$t$时刻,集合$S_{t-1}$中的节点$v\in S_{t-1}$可以将信息以概率$p$传播给它们未被激活的邻居节点$u\notin S_{t-1}$。如果传播成功,即$u$被激活,则$u$将被加入$t$时刻的激活集合$S_t$。集合$S_t$包含$u\notin S_{t-1}$中的所有节点,以及在$t$时刻被激活的所有节点。

不同节点的激活过程相互独立,互不影响。已被激活的节点将永远处于激活状态,未被激活的节点不具备记忆性,下一次有同样的概率$p$被激活。

当网络中没有新的节点被激活时,传播过程停止。

需要注意的是,在本问题中给出的独立级联模型与传统的独立级联模型存在差别。在传统的独立级联模型中,在$t$时刻若节点$v$以概率$p$试图激活节点$u$失败,那么在之后的时刻中,节点$v$不能再激活$u$,即这种激活行为被抛弃\upcite{fudan}。但在本问题中给出的独立级联模型中,“未被激活的节点不具备记忆性,下一次有同样的概率$p$被激活”,即在节点$v$试图激活节点$u$失败之后的时刻中,节点$v$仍可以以概率$p$激活节点$u$。

\subsection{传播效果优化策略}
\subsubsection{基于度数的节点选择策略}
中心度是分析社会网络的一个最重要的和常用的概念工具之一。它是关于行动者在社会网络中的中心性位置的测量概念,反映的是行动者在社会网络结构中的位置或优势的差异。在一个社会网络中,某节点度数最高,该点就居于中心位置,这表明该点所对应的行动者为此网络中的中心人物即最具影响力的人物 \upcite{Lin}。在社会网络和其它网络中,以度数递减的顺序选择$k$个最大度数节点的启发式节点选择策略,是长期以来的一个标准方法,在社会科学中被称为“度中心性”。此方法的一个缺点就是静态选择初始节点。没有考虑影响的扩散过程。不能保证最终影响范围最优。

\subsubsection{集合覆盖贪心算法}
集合覆盖贪心算法\upcite{Estevez}每次选择最高“uncovered”度数的节点,一旦一个节点被选中,它的所有邻居节点被标记为“covered”,这个过程一直持续$k$步。它选择覆盖范围最大的节点,但是在影响最大化的约束下,覆盖并不等于激活,所以其实验结果并不好。


\subsubsection{爬山贪心算法}
为了找到模型中要求的初始扩散集合$S_k$,一个简单有效的策略是每一步根据算法的标准确定初始集合中的一个节点,直到找到$k$个(预定义)节点。

(1)$S_0=\varnothing $;

(2)$I(S_i)$:集合$S_i$扩散后已激活节点的集合;

(3)$m(u|S_i)=|I(S_i\cup \{u\})|-|I(S_i)|$:节点$u$相对于集合$S_i$的边际影响范围。

尽管爬山贪心算法能够在$1 - 1/e$的因子内近似最优值,但是其缺点也十分明显,每一步都要计算所有未激活节点$u$的边际影响$m(u|S_i)$,这导致了该算法运行十分耗时。文献\upcite{Chen}利用IC模型的子模特性给出了自然爬山贪心算法的改进方法CELF算法。利用子模特性,在每一步选择初始种子节点时,大量节点的增量影响不需要被重新计算。CELF算法缩短了自然爬山贪心算法的时间。



\section{问题描述}
给定一个弱连通的有向网络,包含1377个节点(network\_nodes.txt)和2279条边(network\_edgs.txt)。网络如图 \ref{problemnetworkgraph}所示。

\begin{figure}[h!]
	\centering
	\includegraphics[width=0.8\linewidth]{problemnetworkgraph.png}
	\caption{本问题社会网络示意图}
	\label{problemnetworkgraph}
\end{figure}

此次的信息传播优化问题描述如下:

假设信息传播过程可以用独立级联模型来描述,其传播概率$p=20\%$,传播时间$T=8$s,在一给定弱连通的有向网络(1377个节点,2279条边),问:
\begin{enumerate}
	\item 如何选择10个节点,使得信息的传播范围最广?
	\item 如果希望信息的传播能覆盖800个以上的节点,则最少应选择哪些用户作为传播的起始起点?
\end{enumerate}



\section{社会网络信息传播模型}
\subsection{基于概率的信息传递过程}
在本问题中,每个节点$v_i$是以概率$p$激活相邻节点$u_j$,即在激活过程中存在不确定性。若运算过程中存在随机过程,必然会导致在不同的随机种子下,得出不同的运算结果。这将直接影响每个节点“影响力”大小的计算,并最终影响最优初始节点的选择决策。为克服上述问题,本文不在运算过程中引入随机过程,而是完全从概率的角度,计算每个节点$v_i$在规定传播时间$T$和传播概率$p$条件下,最终被激活的概率$P_{v_i}$。

\begin{figure}[h!]
	\centering
	\includegraphics[width=0.3\linewidth]{probability1.png}
	\caption{只考虑有向边$(v,u)$}
	\label{probability1}
\end{figure}

定义$P_{v,t}$为节点$v$在$t$时刻已被激活的概率。若只考虑有向边$(v,u)$如图 \ref{probability1},根据条件概率公式 \ref{condiprob}(其中A为节点$u$被激活,B为节点$v$被激活),节点$u$在时间$t$被节点$v$激活的概率$P_{u,t}$的计算公式如公式\ref{probdisequ1}\upcite{statistics}:
\begin{equation}\label{condiprob}
P(B)=P(AB)=P(A)P(B|A)
\end{equation}
\begin{equation}\label{probdisequ1}
P_{u,t}=P_{v,t-1} \cdot p
\end{equation}
其中,$p$为传播概率,$P_{v,t-1}$为节点$v$在$t-1$时刻已被激活的概率。

\begin{figure}[h!]
	\centering
	\subfigure[多条边指向同一节点]{
		\label{probgraph2} %% 第一幅图的标签
		\includegraphics[width=0.25\textwidth]{probdistribution.png}}
	\hspace{0.1in} \subfigure[多级指向]{
		\label{probgraph3} %% 第二幅图的标签
		\includegraphics[width=0.26\textwidth]{probability3.png}}
	\caption{无向图和有向图}
	\label{概率传播特殊情况} %% label for entire figure
\end{figure}



但在实际情况中,网络不是只有一条有向边,而是存在多条边指向同一个节点。同时,因为节点$v$有多次激活$u$的机会,所以在$t$时刻时,$P_{u,t-1}$可能大于零。此时便无法通过公式\ref{条件概率}和\ref{概率传播公式1}简单求得,以图\ref{probgraph2}为例,综合使用概率论的有关知识,可以得到节点$u$在时间$t$已被前节点激活的概率$P_{u,t}$的计算公式如公式\ref{概率传播公式2}:
\begin{equation}\label{概率传播公式2}
P_{u,t}=1-(1-P_{u,t-1})[(1-p \cdot P_{v_0,t-1})(1-p \cdot P_{v_1,t-1})(1-p \cdot P_{v_2,t-1})]
\end{equation}
其中,$p$为传播概率,$P_{v_i,t-1}$为节点$v_i$在$t-1$时刻已被激活的概率,$P_{u,t-1}$为节点$u$在$t-1$时刻已被激活的概率。

为方便程序计算每个节点的影响因子,本文编写的程序采用前向传播计算方式,及遍历前节点$v_i$,对每个前节点$v_i$计算其后节点$u_i$的激活概率,此时存在一个问题:以图\ref{概率图2}为例,在利用节点$v_1$计算$P_{u,t}$时,$P_{u,t}$已在利用节点$v_0$计算时计算过一次。此时在计算$P_{u,t}$要考虑已存在的$P_{u,t}$。另外还需注意,以图\ref{概率图3}为例,在计算$P_{u,t}$时只能使用$P_{v_1,t-1}$,而不能使用$P_{v_1,t}$,因为在$\Delta t=1$内只能传播一次。综上,我们给出了修正后的概率传播迭代计算公式如公式 \ref{概率传播}:
\begin{equation}\label{概率传播}
P_{u,t}=1-(1-P_{u,t})(1-p \cdot P_{v_i,t-1})
\end{equation}
注意只有遍历了所有的$P_{v_i,t-1}(v_i \in V)$后,计算得到的$P_{u,t}$才是最终的$P_{u,t}$。


\subsection{基于概率的独立级联模型}
基于上述信息传播过程概率模型,我们提出基于概率的独立级联模型:

(1)选定初始节点集合$V_{init}$;

(2)在$t=0$时刻,信息从选定的初始节点$v_{init}(v_{init}\in V_{init})$开始第一次传播。这些初始节点被认为是处于激活状态,即$P_{v_{init},0}=1$,$P_{v_i,0}=0(v_i\notin V_{init})$;

(3)在$t=1$时刻,对于每一个节点$v_i$,结合传播概率$p$计算其信息传播方向(与边的指向一致)指向的所有节点$u_j$的$P_{u_j,1}$;

\ldots \ldots

(4)在$t$时刻,对于每一个节点$v_i$,利用$P_{v_i,t-1}$和传播概率$p$计算其信息传播方向(与边的指向一致)指向的所有节点$u_j$的$P_{u_j,t}$;

\ldots \ldots

(5)重复(3)直到$t=T$,信息传播终止,得到最终对于全部$v_i$的激活概率$P_{v_i}$。
~\\

仔细观察上述基于概率的独立级联模型可以发现以下特点:
\begin{itemize}
	\item 除了初始节点$v_{init}$的$P_{v_{init}}=1$外,其它节点的$P_{v_{i},t}$始终在$[0,1)$范围内,不可能达到1;
	\item 对于任意$t\geqslant 1$,$P_{v_{i},t}$始终$\geqslant P_{v_{i},t-1}$。
\end{itemize}

计算各节点$v_{i}$最终激活概率$P_{v_{i}}$的算法如算法 \ref{最终激活概率}:
\begin{algorithm}[h!]
	\caption{计算各节点$v_{i}$的最终激活概率$P_{v_{i}}$}
	\label{最终激活概率}	
	Initialization: 选定初始节点集合$V_{init}$;\\
	$P_{v_{init},0}=1(v_{init}\in V_{init})$,$P_{v_i,0}=0(v_i\notin V_{init})$,	$t=1$;\\
	\While{$t \leqslant T$}
	{
		\For {$v_i \in V$}
		{
		  \For {$ u_j \in U_{v_i}$}
		  {
		  	// $U_{v_i}$是节点$v_i$可以指向的节点的集合\\
		  	$P_{u_j,t}=1-(1-P_{u_j,t})(1-p \cdot P_{v_i,t-1})$;
		  }
		}
	}
	\For {$v_i \in V$}
	{
		$P_{v_i}=P_{v_i,T}$
	}
\end{algorithm}




\section{传播效果优化模型}
本问题要求在规定时间内,选择定量的初始节点达到尽可能大的影响范围;或是在确定影响范围的前提下,选择尽可能少的初始节点,即归纳为一个对社会网络传播效果的优化问题。本问题的网络结构包含1377个节点和2279条边,网络复杂度不算很高。为了在可接受的运算时间内得到尽可能精确的结果,我们采用CELF算法优化的爬山贪心算法来对本问题进行分析和求解。
\subsection{CELF算法优化的爬山贪心算法}
\subsubsection{自然爬山贪心算法}
为了找到问题中要求的最优初始节点集合$V_{init}$,自然爬山贪心算法的思想是每一步根据算法的标准(本问题中即为节点的边际影响因子)确定初始集合中的一个节点,直到找到问题要求的$k$个节点,如算法 \ref{自然爬山贪心算法}所示。
\begin{algorithm}[h!]
	\caption{自然爬山贪心算法}
	\label{自然爬山贪心算法}	
	Initialization: $V_{init}=\varnothing , i=0, V_{alternative}=V $;\\
	\While{$i < k$}
	{
		\For {$v_i \in V_{alternative}$}
		{
			计算$m(v_i|V_{init})=|I(V_{init}\cup \{v_i\})|-|I(V_{init})|$;\\
			//$I(V_{i})$是集合$V_{i}$扩散后的传播影响效果,$m(v_i|V_{init})$是节点$v_i$相对于集合$V_{init}$的边际影响效果;
		}
		选择最大的$m(v_i|V_{init})$对应的$v_i$,将其加入$V_{init}$,并将其从$V_{alternative}$移除;
		$i=i+1$;
	}
	得到最终的最优初始节点集合$V_{init}$。
\end{algorithm}

\subsubsection{IC模型的子模性质}
影响力传播模型中的独立级联模型(independent cascading model,IC模型),影响力传播过程中,种子的影响力具备子模性(submodularity),即种子的边际影响力增量会呈现递减趋势\upcite{云南大学}。

\begin{figure}[h!]
	\centering
	\includegraphics[width=0.9\linewidth]{子模.png}
	\caption{子模性质示意图}
	\label{子模}
\end{figure}

以图 \ref{子模}为例,针对本问题,假设在选择第二个节点($k=1$)时,计算得节点$v_1, v_2, v_3$的边际影响效果(影响因子)分别为10, 5, 3。那么可以证明:在选择第三个节点($k=2$)时,计算得节点$v_1, v_2, v_3$的边际影响效果(影响因子)一定分别$\leqslant10$, $\leqslant5$, $\leqslant3$,即种子的边际影响力增量会呈现递减趋势。子模性的证明过程在本文中不详细探讨。

\subsubsection{CELF算法}
CELF算法利用了IC模型的子模性质,大大减少了自然爬山贪心算法的计算量,减少了运算时间。

以图 \ref{CELF}为例,在选择最优节点时,自然爬山贪心算法将遍历整个$V_{alternative}$,计算所有可选节点的边际影响力,并从中选出最大的节点加入最优初始节点集合之中。

而在CELF算法中,根据IC模型条件下,节点的$\Delta infs$符合子模性,于是在A加入$V_{init}$后,在下一轮计算各个节点的边际影响效果$\Delta infs$时,若计算出B的$\Delta infs$大于或等于上一轮中比它小且最接近它的那个节点(C)在上一轮中的$\Delta infs$,那么这一轮就可以直接把B加入到$V_{init}$当中,而不用计算后面C,D,E...等节点的$\Delta infs$,因为它们在这一轮的$\Delta infs$必定比上一轮自己的$\Delta infs$小。

\begin{figure}[h!]
	\centering
	\includegraphics[width=0.7\linewidth]{CELF.png}
	\caption{CELF算法示意图}
	\label{CELF}
\end{figure}

如图\ref{CELF}所示,如果$\Delta b \geqslant 8$,那么B就是$\Delta infs$最大的节点,B可直接加入$V_{init}$,不用再计算$\Delta c$,$\Delta d$,$\Delta e$等等,因为依据子模性(即边际递减规律),种子集$V_{init}$加入了A之后,$\Delta c$必定小于等于8,$\Delta d$必定小于等于7,$\Delta e$必定小于等于5。CELF算法因此大大降低了自然爬山贪心算法的计算量。

若这一轮B的$\Delta infs$没有大于或者等于8,则逐个算出每个节点的$\Delta infs$,选出其中$\Delta infs$最大的节点加入$V_{init}$,然后重复上述过程。

\subsection{影响因子的定义}
在爬山贪心算法中,一个很重要的步骤是计算每个节点的边际影响力,即该节点的影响因子。而如何定义该影响因子,对问题求解的精度有着极大的影响。在很多算法中,某节点的影响因子定义为该节点可以指向的边际节点数目,即$I(V_{i})$等于$V_{i}$中的所有节点能指向的全部节点数目。但显然,这是不准确的,因为该定义仅仅考虑了静态状态下节点的影响力,即只传播一次的情况下节点能传播的范围。但在本问题中,传播过程长达数次,也就意味着,一轮指向的节点所能指向的节点数目同样对优化结果有着极大影响,而每个节点在数轮传播后所能指向的节点数目难以在静态状态下进行评估。为了让我们的模型算法对问题的求解更加精确,更好地优化网络传播影响效果,我们对节点$v_i$的影响因子$Factor_{v_i}$进行如下定义:
\begin{displaymath}
Factor_{v_i} = m(v_i|V_{init})=|I(V_{init}\cup \{v_i\})|-|I(V_{init})|
\end{displaymath}
其中,$I(V_{i})$等于:
\begin{displaymath}
I(V_{i}) = E(Amount\ of\ activated\ nodes)=\sum_{i=1}^N{P_{v_i}}
\end{displaymath}
$N$为网络节点总数,即$I(V_{i})$等于当初始激活节点为$V_i$,以概率$p$传播时间$T$后,被激活的节点总数的期望。所有节点的最终激活概率由本文第4部分中的社会网络信息传播模型在给定初始激活节点的条件下求得。

这也就意味着,在每次计算节点的影响因子时,我们都需要计算整个网络所有节点的最终激活概率。这会导致算法的时间复杂度较高,但鉴于本问题的网络结构不算十分复杂,所以计算耗时还在可接受范围内。而这种定义可以大大地提升问题求解的精度和优化的效果。

\subsection{求解本优化问题的最终模型算法}
综上所述,为求解本文中的社会网络信息传播优化问题,我们采用了CELF优化的爬山贪心算法。对于爬山贪心算法中节点的影响力,我们定义其为将该节点加入初始激活节点后,经过时间$T$的传播后,整个网络中被激活节点总数的期望值。而节点的影响因子为节点的边际影响力,即将该节点加入初始激活节点后,整个网络中被激活节点总数的期望值的提高值。

整个算法如算法 \ref{最终算法}所示。

\begin{algorithm}[h!]
	\caption{求解本优化问题的最终算法}
	\label{最终算法}	
	Initialization: $V_{init}=\varnothing , i=0, V_{alternative}=V $;\\
	\For {$v_i \in V_{alternative}$}
	{
		计算$Factor_{v_i}=m(v_i|V_{init})=|I(V_{init}\cup \{v_i\})|-|I(V_{init})|$;\\
		//$I(V_{i})$是集合$V_{i}$作为初始激活节点传播时间$T$后整个网络中被激活节点总数的期望;
	}
	选择最大的$Factor_{v_i}$对应的$v_i$,将其加入$V_{init}$,并将其从$V_{alternative}$移除;\\
	$i=i+1$; \\
	\While{$i < k$}
	{
		选择之前第二大的$Factor_{v_i}$对应的$v_i$,计算其新一轮的$Factor_{v_i}^{new}$ \\
		\If{$Factor_{v_i}^{new} \geqslant Factor_{v_j}(\forall v_j \in V_{alternative}, j \neq i)$}
		{
			将$v_i$其加入$V_{init}$,并将其从$V_{alternative}$移除;
		}
		\Else
		{
			\For {$v_i \in V_{alternative}$}
			{
				计算$Factor_{v_i}=m(v_i|V_{init})=|I(V_{init}\cup \{v_i\})|-|I(V_{init})|$;\\
				//$I(V_{i})$是集合$V_{i}$作为初始激活节点传播时间$T$后整个网络中被激活节点总数的期望值;
			}
			选择最大的$Factor_{v_i}$对应的$v_i$,将其加入$V_{init}$,并将其从$V_{alternative}$移除;
		}
		$i=i+1$;
	
	}
	得到最终的最优初始节点集合$V_{init}$。
\end{algorithm}



\section{实验结果与分析}
\subsection{小规模网络的验证}
由于本问题中的网络结构对于人的想象力来说较为复杂,难以直观想象,在这个网络结构上计算得结果难以确定其准确性。为了检验本文提出的模型和算法的有效性和准确性,我们首先自己建立了一个小规模的社会信息网络结构,如图\ref{小网络}所示。在此小规模网络上,我们根据本问题的相关条件,测试检验我们提出的模型和算法。

\begin{figure}[h!]
	\centering
	\includegraphics[width=0.4\linewidth]{smallnet.png}
	\caption{自建小规模网络示意图}
	\label{小网络}
\end{figure}

\subsubsection{信息传播模型的检验}
首先,我们应用图\ref{小网络}所示小规模网络来检验我们提出的基于概率的信息传播模型的准确性。假设给定初始激活节点为节点2和节点3,传播概率$p=0.2$分别计算经过不同的传播时间t后各个节点的激活概率,如图\ref{不同传播时间的节点激活概率}所示。

\begin{figure}[h!]
	\centering
	\subfigure[$t=1$]{
		\label{t=1} %% 第一幅图的标签
		\includegraphics[width=0.3\textwidth]{t=1.png}}
	\hspace{0.5in} \subfigure[$t=2$]{
		\label{t=2} %% 第二幅图的标签
		\includegraphics[width=0.3\textwidth]{t=2.png}}
	\caption{不同传播时间的节点激活概率}
	\label{不同传播时间的节点激活概率} %% label for entire figure
\end{figure}

初始激活节点为节点2和节点3,当$t=0$时,节点2和节点3的激活概率为1,其它节点的激活概率为0。当$t=1$时(如图\ref{t=1}),传播一次,节点5被节点2尝试激活,$P_{5,1}=P_{2,0}\cdot p=1\times 0.2=0.2$,节点8被节点2和节点3尝试激活,$P_{8,1}=1-(1-P_{2,0}\cdot p)(1-P_{3,0}\cdot p) = 1-(1-1\times 0.2)^2=0.36$,同理可以验证当$t=2$时,本文算法计算的结果(如图\ref{t=2})与检验计算得结果是一致的。由此验证了本文提出的基于概率的信息传播模型的准确性和可靠性。

\subsubsection{传播效果优化模型的检验}
应用本文提出的基于CELF算法优化的爬山贪心算法的传播优化模型,在图\ref{小网络}所示网络中根据先决条件选择最优初始节点。假定初始激活节点数目$N=2$,传播概率$p=0.2$,传播时间$T=2$,本文算法选出的最优初始激活节点为\{节点3,节点5\},整个网络中被激活节点总数的期望值为3.19个。

同时我们测试了其它初始激活节点选择的最终各节点激活概率,如图\ref{不同初始激活节点传播后的各节点最终激活概率}所示,其中当初始激活节点为\{节点2,节点3\}时,最终整个网络中被激活节点总数的期望值为3.06个;当初始激活节点为\{节点3,节点8\}时,最终整个网络中被激活节点总数的期望值为2.76个,均小于本文算法选择的最优初始激活节点的传播效果。由此检验了本文提出的传播效果优化模型的优化效果。

\begin{figure}[h!]
	\centering
	\subfigure[\{3,5\}]{
		\label{35} %% 第一幅图的标签
		\includegraphics[width=0.25\textwidth]{35.png}}
	\hspace{0.3in} \subfigure[\{2,3\}]{
		\label{23} %% 第二幅图的标签
		\includegraphics[width=0.25\textwidth]{23.png}}
	\hspace{0.3in} \subfigure[\{3,8\}]{
		\label{38} %% 第二幅图的标签
		\includegraphics[width=0.25\textwidth]{38.png}}
	\caption{不同初始激活节点传播后的各节点最终激活概率}
	\label{不同初始激活节点传播后的各节点最终激活概率} %% label for entire figure
\end{figure}

需要注意的是,在本条件下,选择\{节点3,节点5\}和选择\{节点3,节点6\}作为初始激活节点,最终整个网络中被激活节点总数的期望值是相等的,即优化问题可能存在多个最优解。而本文模型算法只能求出其中一个最优解而无法求出全部最优解。

另外我们发现,不同初始节点数目,不同传播时间,不同传播概率对最优初始节点选择策略都会产生影响,下面我们继续以图\ref{小网络}来对这些影响进行讨论和分析。

\begin{table}[h!]
	\centering
	\begin{tabular}{|m{2cm}<{\centering}|m{3cm}<{\centering}|m{3cm}<{\centering}|m{3cm}<{\centering}|}
		\hline
		\diagbox{$\ \ T\ \ $}{$V_{init}$}{$\ \ p\ \ $} & $p=0.2$ & $p=0.5$  &$p=1$ \\
		\hline
		$T=1$ & \{2,6\} & \{2,6\} &\{2,6\}\\
		\hline
		$T=3$ & \{3,5\} & \{3,5\} &\{2,1\}\\
		\hline
		$T=8$ & \{3,6\} & \{3,1\} &\{3,1\}\\
		\hline
		$T=15$ & \{3,1\} & \{3,1\} &\{3,1\}\\
		\hline
		$T=30$ & \{3,1\} & \{3,1\} &\{3,1\}\\
		\hline
	\end{tabular}
	\caption{最优初始激活节点选择结果($N=2$)}
	\label{N=2}
\end{table}
\begin{table}[h!]
	\centering
	\begin{tabular}{|m{2cm}<{\centering}|m{3cm}<{\centering}|m{3cm}<{\centering}|m{3cm}<{\centering}|}
		\hline
		\diagbox{$\ \ T\ \ $}{$V_{init}$}{$\ \ p\ \ $} & $p=0.2$ & $p=0.5$  &$p=1$ \\
		\hline
		$T=1$ & \{2,6,3\} & \{2,6,4\} &\{2,6,1\}\\
		\hline
		$T=3$ & \{3,5,1\} & \{3,5,1\} &\{2,1,3\}\\
		\hline
		$T=8$ & \{3,6,1\} & \{3,1,4\} &\{3,1,4\}\\
		\hline
		$T=15$ & \{3,1,4\} & \{3,1,4\} &\{3,1,4\}\\
		\hline
		$T=30$ & \{3,1,4\} & \{3,1,4\} &\{3,1,4\}\\
		\hline
	\end{tabular}
	\caption{最优初始激活节点选择结果($N=3$)}
	\label{N=3}
\end{table}

表\ref{N=2}和表\ref{N=3}分别记录了初始激活节点数目为2和3时,在不同的传播时间$T$和不同的传播概率$p$下,选择的最优初始激活节点的结果。

首先,将两张表进行对比,我们可以发现,在相同的传播时间$T$和相同的传播概率$p$的条件下,$N=3$中选择的3个节点的前两个与$N=2$中选择的相同。显然,这是因为我们采用的是贪心算法,每一步都是取局部最优,即$N=3$中选择的最优节点是在$N=3$中选择的最优节点的基础上又再选择了一个节点。

单独观察表\ref{N=2}可以发现,对于不同的传播时间$T$和不同的传播概率$p$,算法所选择的最优初始节点时不同的。以不同的传播时间$T$为例,取两个极端的例子:$T=1$和$T=30$。当$T=1$时,因为只传播一次,那么显然应该选择覆盖范围大的点作为初始节点,即不选择节点1和节点4。节点2和3都可以指向两个节点,那么应该选择其中之一,但不应两个都选,因为若两个都选的话,节点3给节点2的传播相当于被浪费了,因为节点2已经被激活了。同理,第二个点应该选择自身未被激活且不能被第一个点激活且还能给传播至其它节点的节点,这样能最大化所有传播路径的效果。由此选择节点6,满足上述要求。最终选择节点2和节点6是合理的。

而当$T=30$时,此时传播时间很大,传播了很多次。因为节点3可以直接或间接指向节点2,5,6,7,8,9,当$T$无穷大时,若选择节点3作为初始激活节点,节点2,5,6,7,8,9的被激活概率都接近于1。此时,如果再在节点2,5,6,7,8,9中选择第二个节点,那就相当于浪费了,所以在节点1和节点4中选择一个,达到传播效果最大化。最终选择节点3和节点1是合理的。

综上所述,不同初始节点数目,不同传播时间,不同传播概率对最优初始节点选择策略都会产生影响,尤其是不同传播时间$T$极大地影响了模型和算法求解的准确性,不迭代完整个时间$T$很难评估初始激活节点的最终传播效果。由此也证明了我们所建立的较为精确的基于概率的传播模型和优化算法的必要性。

\subsection{对本问题的求解和分析}
\subsubsection{问题一:选择10个初始激活节点,优化传播范围}
应用本文提出的社会信息传播模型和传播效果优化模型,设定传播时间$T=8$,传播概率$p=0.2$,在本问题的网络结构中,对最终被激活节点数目的期望进行优化。算法选择的10个最优初始激活节点列在表\ref{问题一结果}中,最终整个网络中被激活节点总数的期望值为80.97个。

\begin{table}[h!]
	\centering
	\begin{tabular}{|c|c|c|c|c|}
		\hline
		1667516751 & 1253383617 & 1722445411 & 1518610444 & 1497554923  \\
		\hline
		1657301617 & 1080547893 & 1623666844 & 1644352745 & 1655288664  \\
		\hline
	\end{tabular}
	\caption{问题一:最优初始激活节点}
	\label{问题一结果}
\end{table}

为验证本算法的优化效果,并说明应用本算法的优越性和必要性,我们将其与一种简单算法的优化效果进行比较。该简单算法即将每个节点能指向的节点数目进行排序,并从中选出最大的10个作为10个初始激活节点。两种算法选择的最优初始节点和最终传播效果如表\ref{对比}所示。

该简单算法选择的节点能直接指向的节点数目都在10个以上,最大的达到了16个,本文算法的解中有些点也在其中。但通过表\ref{对比}我们可以很直观地看出,本文提出的算法的优化效果要好于该简单算法。虽然简单算法选择的都是直接指向节点数最多的节点,但这并不代表它们的间接覆盖范围大,而且其中可能还存在很多重复的覆盖范围,所以效果不如本文算法。由此验证了本文所提出的优化算法的优越性。

\begin{table}[h!]
	\centering
	\begin{tabular}{|m{2cm}<{\centering}|m{4cm}<{\centering}|m{4cm}<{\centering}|}
		\hline
		&本文算法 & 简单算法 \\
		\hline
		&1667516751 & 1667516751 \\
		\cline{2-3}
		&1253383617 &1458684275\\
		\cline{2-3}
		&1722445411 &1080547893\\
		\cline{2-3}
		&1518610444 &1009842360\\
		\cline{2-3}
		最优初始&1497554923 &1253383617\\
		\cline{2-3}
		激活节点&1657301617 &1687864635\\
		\cline{2-3}
		&1080547893 &1291285737\\
		\cline{2-3}
		&1623666844 &1683329950\\
		\cline{2-3}
		&1644352745 &15058618\\
		\cline{2-3}
		&1655288664 & 1649591804\\
		\hline
		被激活节点总数的期望 & 80.97个 & 61.69个 \\
		\hline
	\end{tabular}
	\caption{本文算法与一种简单算法传播效果对比}
	\label{对比}
\end{table}

\subsubsection{问题二:传播范围达到800个节点,优化初始节点数目}
应用本文提出的社会信息传播模型和传播效果优化模型,设定传播时间$T=8$,传播概率$p=0.2$,在本问题的网络结构中,选择尽可能少的初始激活节点,使最终被激活节点数目的期望达到800个以上。最终本文算法选择的初始节点数目为557个(如图\ref{557}所示),被激活节点数目的期望随初始激活节点数目的变化曲线如图\ref{result}所示。

从图\ref{result}中可以直观地看出:在前期,初始激活节点数目增加会使最终被激活节点数目的期望大幅增加,随着初始激活节点数目不断越多,被激活节点数目的期望的增幅逐渐减小;可以发现,在初始激活节点数目达到150个左右后,每增加一个初始激活节点,只能使最终被激活节点数目的期望增加1个左右。

\begin{figure}[h!]
	\centering
	\includegraphics[width=0.8\linewidth]{result.png}
	\caption{激活节点数目变化曲线}
	\label{result}
\end{figure}

\begin{figure}[h!]
	\centering
	\includegraphics[width=0.9\linewidth]{557.png}
	\caption{选择的557个初始激活节点}
	\label{557}
\end{figure}

\newpage
\section{总结}
本文基于社会信息网络传播和独立级联模型,结合概率模型,提出了基于概率的社会信息网络传播模型,使社会信息的网络化传播得到了更准确的建模,使传播时间和传播概率等因素得到更好地描述。对于初始节点优化问题,本文应用CELF算法优化的爬山贪心算法,以最终被激活节点数目的期望的边际值作为影响因子,使爬山贪心算法的结果更加准确,更接近全局最优。对于本问题中的问题一,应用本文中的模型和算法,选择10个最优初始激活节点,最终被激活节点数目的期望可以达到80.97个;对于问题二,应用本文中的模型和算法,最少选择557个初始激活节点,可以使得最终被激活节点数目的期望达到800个以上。

尽管本文提出的算法的准确度比较好,但由于爬山贪心算法本质上在每次选择的时候都计算一遍所有的节点的影响因子,所以时间复杂度较高。本问题中的网络规模不算特别大,故可以应用本算法,但如果网络规模非常大时,应用本算法的耗时将无法接受。在保证算法准确度的情况下如何降低算法的时间复杂度,将是需要进一步讨论的问题。




\newpage
\begin{thebibliography}{}
	\bibitem{Richardson}Richardson M, DomingosP , Glance N. Mining knowledgesh
	aring sites for Viral Marketing //Proceedings of the 8th
	ACM SIGKDD International Conference on Knowledge Discovery
	and Data Mining. Edmonton, Canada, 2002: 61-70
	
	\bibitem{Kempe}Kempe D, Kleinberg J, Tardos E. Maximizing the spread of
	influence in a social network //Proceedings of the 9th ACM
	SIGKDD International Conference on Knowledge Discovery
	and Data Mining. Washington, USA, 2003: 137-146
	
	\bibitem{Lin}Lin Ju Ren. Social Network Analysis: Theory, Methods and
	Applications. Beijing: Beijing Normal University Press, 2009
	
	\bibitem{Estevez}Estevez Pablo A, Vera Pablo, Saito Kazumi. Selecting the
	most influential nodes in social networks //Proceedings of the
	International Joint Conference on Neural Networks. Orlando,
	Florida, USA, 2007: 2397-2402
	
	\bibitem{Chen}Chen Wei, Wang Ya-Jun, Yang Si-Yu. Efficient influence
	maximization in social networks //Proceedings of the 15th
	ACM SIGKDD International Conference on Knowledge Discovery
	and Data Mining. Paris, France, 2009: 199-208
	
	\bibitem{statistics}王宁. 概率统计与随机过程[M]. 西安:西安交通大学出版社, 2018: 14.
	
	\bibitem{fudan}田家堂,王轶彤,冯小军. 一种新型的社会网络影响最大化算法[J]. 计算机学报, 2011, (10).

	\bibitem{yunnan}黄玮鹏. 基于扩展独立级联模型的竞争影响最大化传播[D]. 昆明: 云南大学, 2015.

\end{thebibliography}

\clearpage

\end{document}
