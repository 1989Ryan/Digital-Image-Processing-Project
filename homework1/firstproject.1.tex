\documentclass[UTF8]{ctexart}
\usepackage{geometry}
\usepackage{mathrsfs}
\usepackage{graphicx}
\usepackage{multirow}
\usepackage{amsmath}
\usepackage{amssymb}
\usepackage{array}
\usepackage{subfigure}
\usepackage{algorithm2e}
\usepackage{setspace}
\usepackage{diagbox}
\usepackage{bmpsize}
\geometry{a4paper,scale=0.75}
\newcommand{\xiaosi}{\fontsize{11.6pt}{\baselineskip}\selectfont}
\newcommand{\sihao}{\fontsize{12.1pt}{\baselineskip}\selectfont}
\newcommand{\upcite}[1]{\textsuperscript{\textsuperscript{\cite{#1}}}}
\renewcommand{\baselinestretch}{1.3}

%opening
\title{数字图像处理第一次作业}
\author{赵子瑞 \\ 自动化钱61班}
\date{2019年3月5日}
\begin{document}

\maketitle

\newenvironment{cnabandkey}[2][\sihao 摘要] % 定义中文摘要和关键词环境
{\newcommand{\ckeywords}{#2} %
	\begin{center} \bfseries #1 \end{center} %
	\begin{quotation}
	}{\paragraph{\sihao 关键词:} \textrm{\ckeywords} %
	\end{quotation}
}

\newenvironment{enabandkey}[2][\sihao Abstract] % 定义英文摘要和关键词环境
{\newcommand{\ekeywords}{#2} %
	\begin{center} \bfseries #1 \end{center} %
	\begin{quotation}
	}{\paragraph{\sihao Keywords:} \textrm{\ekeywords} %
	\end{quotation}
}

	\begin{cnabandkey}{bmp图像格式\, 图像处理\, 灰度降级\, 图像均值方差计算\, 图像插值\, 图像旋转\, 图像切割}
        \xiaosi
        \hspace{0.25em}本文是数字图像与视频处理的第一次作业,通过理论学习和代码实践,完成了以下任务:
        
        首先介绍了bmp图像格式,从存储算法,文件格式和像素存储等方面进行介绍,并且按照给定的图像为例进行说明,通过解析图片和存储结构,来得到例图的详细信息。
        
        其次,本文将对lena图片进行一系列操作,对lena的照片进行灰度递减显示,计算均值方差,利用多种方法进行插值,
        同时对lena和elain的图像进行切割、旋转和插值。从而加深我们对图像处理基本知识的掌握
		
	\end{cnabandkey}
	
	\clearpage
	
\begin{spacing}{1.2} %正文部分
\xiaosi\tableofcontents\newpage
\end{spacing}


\xiaosi
\section{项目任务}

本次数字图像与食品处理主要是对课堂上介绍的算法进行实践,进而加深对图像处理的基本知识的掌握情况。

(1)BMP图像格式简介,以$7.bmp$为例说明;

(2)把$lena$ $512 \times 512$的图像灰度从1到8逐级递减显示;

(3)计算$lena$图像的均值方差

(4)把$lena$图像用近邻、双线性和双三次插值法放大到2048x2048;

(5)把$lena$和$elain$图像分别进行水平剪切(参数可设置为1.5,或者自行选择)和旋转30度,并采用用近邻、双线性和双三次插值法放大到$2048\times2048$;

\section{bmp图像简介}
BMP取自位图\textbf{B}it\textbf{M}a\textbf{P}的缩写,也称为DIB(与设备无关的位图),是一种与显示器无关的位图数字图像文件格式。常见于微软视窗和OS/2操作系统,Windows GDI API内部使用的DIB数据结构与 BMP 文件格式几乎相同。

图像通常保存的颜色深度有2(1位)、16(4位)、256(8位)、65536(16位)和1670万(24位)种颜色(其中位是表示每点所用的数据位)。8位图像可以是索引彩色图像外,也可以是灰阶图像。表示透明的alpha通道也可以保存在一个类似于灰阶图像的独立文件中。带有集成的alpha通道的32位版本已经随着Windows XP出现,它在视窗的登录和主题系统中都有使用\upcite{MSdoc}。

从开发人员的角度来看,位图由一组指定或包含以下元素的结构组成:

(1)一个标题,描述创建像素矩形的设备的分辨率,矩形的尺寸,位数组的大小等。

(2)逻辑调色板。

(3)一个位数组,用于定义图像中的像素与逻辑调色板中的条目之间的关系。

位图大小与其包含的图像类型有关。位图图像可以是单色或彩色。在图像中,每个像素对应于位图中的一个或多个位。单色图像具有每像素1比特(bpp)的比率。彩色成像更复杂。位图可以显示的颜色数等于两个每像素位数。因此,256色位图需要8$bpp$($2 ^ 8 = 256$)。

控制面板应用程序是使用位图的应用程序的示例。为桌面选择背景(或壁纸)时,实际上选择了一个位图,系统使用该位图来绘制桌面背景。系统通过在桌面上重复绘制32 x 32像素的图案来创建所选的背景图案。

图\ref{Redbrickbmp}显示了开发人员对$Redbrick.bmp$文件中位图的透视图。它显示了一个调色板数组,一个32 x 32像素的矩形,以及将调色板中的颜色映射到矩形中的像素的索引数组。

\begin{figure}[h!]
	\centering
	\includegraphics[width=0.6\linewidth]{bmpexample.png}
	\caption{Redbrick.bmp文件}
	\label{Redbrickbmp}
\end{figure}

在前面的示例中,使用16种颜色的调色板在VGA显示设备上创建像素矩形。16色调色板需要4位索引; 因此,将调色板颜色映射到像素颜色的数组也由4位索引组成。

\subsection{文件格式}

位图图像文件由若干大小固定(文件头)和大小可变的结构体按一定的顺序构成。典型的BMP图像文件由四部分组成:

(1)位图头文件数据结构,它包含BMP图像文件的类型、显示内容等信息;

(2)位图信息数据结构,它包含有BMP图像的宽、高、压缩方法,以及定义颜色等信息;

(3)调色板,这个部分是可选的,有些位图需要调色板,有些位图,比如真彩色图(24位的BMP)就不需要调色板;

(4)位图数据,这部分的内容根据BMP位图使用的位数不同而不同,在24位图中直接使用RGB,而其他的小于24位的使用调色板中颜色索引值。

表示位图中像素的比特是以行为单位对齐存储的,每一行的大小都向上取整为4字节(32位DWORD)的倍数。如果图像的高度大于1,多个经过填充实现对齐的行就形成了像素数组。

完整存储的一行像素所需的字节数可以通过公式\ref{RowSize}计算:

\begin{equation}\label{RowSize}
    RowSize = [ \frac{BitsPerPixel \cdot ImageWidth + 31}{32} ] \cdot 4
\end{equation}

其中$ImageWidth$以像素为单位

\subsubsection{像素数组}

这部分逐个像素表示图像。每个像素使用一个或者多个字节表示。通常,像素是从下到上、从左到右保存的。但如果使用的不是BITMAPCOREHEADER,那么未压缩的Windows位图还可以从上到下存储,此时图像高度为负值。

每一行的末尾通过填充若干个字节的数据(并不一定为0)使该行的长度为4字节的倍数。像素数组读入内存后,每一行的起始地址必须为4的倍数。这个限制仅针对内存中的像素数组,针对存储时,仅要求每一行的大小为4字节的倍数,对文件的偏移没有限制。
例如:对于24位色的位图,如果它的宽度为1像素,那么除了每一行的数据(蓝、绿、红)需要占3字节外,还会填充1字节;而如果宽为2像素,则需要2字节的填充;宽为3像素时,需要3字节填充;宽为4像素时则不需要填充。

图像相同的条件下,位图图像文件通常比使用其它压缩算法的图像文件大很多。

\subsubsection{压缩}

索引色图像可以使用4位或8位RLE或霍夫曼1D算法压缩。
OS/2 BITMAPCOREHEADER2 24位色图像则可以使用24位RLE算法压缩。
16位色与32位色图像始终为未压缩数据。
如果需要,任何色深的图像都可以以未压缩形式存储。

\subsubsection{像素存储}

无论是磁盘上的位图文件还是内存中的位图图像,像素都由一组位(英语:bit)表示。

(1)每像素占1位(色深为1位,1bpp)的格式支持2种不同颜色。像素值直接对应一个位的值,最左像素对应第一个字节的最高位。使用该位的值用来对色表的索引:为0表示色表中的第一项,为1表示色表中的第二项(即最后一项)。

(2)每像素占2位(色深为2位,2bpp)的格式支持4种不同颜色。每个字节对应4个像素,最左像素为最高的两位(仅在Windows CE中有效)。需要使用像素值来对一张含有4个颜色值的色表进行索引。

(3)每像素占4位(色深为4位,4bpp)的格式支持16种不同的颜色。每个字节对应2个像素,最左像素为最高的四位。需要使用像素值来对一张含有16个颜色值的色表进行索引。

(4)每像素占8位(色深为8位,8bpp)的格式支持256种不同的颜色。每个字节对应1个像素。需要使用像素值来对一张含有256个颜色值的色表进行索引。

(5)每像素占16位(色深为16位,16bpp)的格式支持65536种不同的颜色,每2个字节(byte)对应一个像素。该像素的不透明度(英语:alpha)、红、绿、蓝采样值即存储在该2个字节中。

(6)每像素占24位(色深为24位,24bpp)的格式支持16777216种不同的颜色,每3个字节对应一个像素。

(7)每像素占32位(色深为32位,32bpp)的格式支持4294967296种不同的颜色,每4个字节对应一个像素。

为了区分一个颜色值中的哪些位表示哪种采样值,DIB头给出了一套默认规则,同时也允许使用BITFIELDS将某组位指定为像素中的某个通道。

\subsection{bmp图像举例}

利用Python和OpenCV将图片进行解析,得到了一个$7\times7\times3$的一个张量,这反映了这个图像是$7\times7$的图像,拥有三个颜色(R,G,B)的通道,其中三个通道的数据完全相同,所以我们可以用下面的矩阵表示。
\[\left( \begin{array}{ccccccc}
82  & 82  & 73  & 59  & 55  & 80  & 90 \\
97  & 89  & 90  & 95  & 71  & 40  & 69 \\
104 & 71  & 63  & 105 & 93  & 76  & 42 \\
88  & 75  & 85  & 101 & 90  & 91  & 70 \\
97  & 92  & 91  & 99  & 72  & 71  & 82 \\
98  & 101 & 102 & 86  & 69  & 71  & 95 \\
103 & 99  & 100 & 84  & 86  & 98  & 98 
\end{array} \right)\]

同时直接读取$7.bmp$的图像结构信息,可以得到以下信息:

(1)1,2位,为``BM'',表示是Windows位图。

(2)一个4字节整数,换算成十进制为1134,表示位图大小; 

(3)一个4字节整数,保留位,始终为0; 

(4)一个4字节整数,换算为十进制是1078,表示实际图像的偏移量; 

(5)一个4字节整数,换算为十进制是40,表示Header的字节数; 

(6)一个4字节整数,十进制为7,表示图像宽度; 

(7)一个4字节整数,十进制为7,表示图像高度; 

(8)一个2字节整数,始终为1; 

(9)一个2字节整数,十进制为8,表示颜色位数。

由此,可以得到这个BMP图片的基本信息。

\section{bmp图像处理}

本节将阐述对我对图像$lena.bmp$和$elain.bmp$进行相应的处理的过程和算法。这里为了未来的作业的处理方便,我创建了一个用来处理数字图像的工具包$``CV python toolbox''$,这里包含了所有作业中布置的任务,包括了图像的读取,图像信息的分析,以及
其他各种计算。同时也调用了部分opencv的库,本着学习和实践的目的,工具包中的核心算法和处理过程均为自行实现。这里采用了Python作为主体语言,主要是考虑到为了方便未来可能会采用深度学习的相关工具包。

\subsection{$lena$图像的灰度递减}

首先我将对$lena.bmp$图像进行灰度递减的操作。这里将$lena.bmp$的灰度值的bit位进行逐级递减的操作,从8位逐级降到1位,经过灰度降级操作,灰度的层级丰富程度也由256种降到了2种。我们的处理结果可以如下图\ref{greyscale_reduce_result}所示。

\begin{figure}[h!]
	\centering
	\subfigure[8bit图像]{
		\label{8bitlena} %%first figure label
		\includegraphics[width = 0.2\textwidth]{lena.png}}
	\hspace{0.1in} \subfigure[7bit图像]{
		\label{7bitlena} %%second figure label
		\includegraphics[width = 0.2\textwidth]{7bitlena.png}}
	\hspace{0.1in} \subfigure[6bit图像]{
		\label{6bitlena} %%second figure label
		\includegraphics[width = 0.2\textwidth]{6bitlena.png}}
	\hspace{0.1in} \subfigure[5bit图像]{
		\label{5bitlena} %%second figure label
		\includegraphics[width = 0.2\textwidth]{5bitlena.png}}
	\newline \subfigure[4bit图像]{
		\label{4bitlena} %%second figure label
		\includegraphics[width = 0.2\textwidth]{4bitlena.png}}
	\hspace{0.1in} \subfigure[3bit图像]{
		\label{3bitlena} %%second figure label
		\includegraphics[width = 0.2\textwidth]{3bitlena.png}}
	\hspace{0.1in} \subfigure[2bit图像]{
		\label{2bitlena} %%second figure label
		\includegraphics[width = 0.2\textwidth]{2bitlena.png}}
	\hspace{0.1in} \subfigure[1bit图像]{
		\label{1bitlena} %%second figure label
		\includegraphics[width = 0.2\textwidth]{1bitlena.png}}	
	\caption{图像灰度递减结果} 
	\label{greyscale_reduce_result} %%label for entire figure
\end{figure}

在这里我们运用了比较简单的降维方式,采用了手写函数实现的方式,这里对算法\ref{grayscale_reduce_algorithm}进行简要介绍。

\begin{algorithm}[h!]
	\caption{灰度降维}
	\label{grayscale_reduce_algorithm}
	Initialization: Read image information; \\
	Get Paramter: $Greyscale\_Reduce\_Index$; \\
	\While{$i \leqslant Width\_of\_image$}
	{
		\While{$j \leqslant Height\_of\_image$}
		{
			\While{$k \leqslant Color\_Tunnel$}
			{
				$Img\_Pixel_{i,j,k} = \biggl\lfloor\frac{Img\_Pixel_{i,j,k}}{Greyscale\_Reduce\_Index}\biggr\rfloor $ //将像素值变为灰度降维后的位数值 \\
				$Img\_Pixel_{i,j,k} = Img\_Pixel_{i,j,k} \cdot  255 / \biggl\lfloor\frac{255}{Greyscale\_Reduce\_Index}\biggr\rfloor  $ //将降维后的像 \newline //素位数值映射到0-255之间
			}
		}
	}
\end{algorithm}

\subsection{$lena$图像的均值方差}

图像的均值方差计算很容易,通过numpy的矩阵运算方法,可以得到$lena.bmp$和$elain.bmp$的图像的均值和方差,计算结果如表\ref{mean_var}所示。

\begin{table}[h!]
	\centering
	\begin{tabular}{|m{2cm}<{\centering}|m{3cm}<{\centering}|m{3cm}<{\centering}|}
		\hline
		$image$ & $mean$ & $variance$ \\
		\hline
		$lena$ & 99.0512 & 2796.0318 \\
		\hline
		$elain$ & 135.3779 & 2120.6471\\
		\hline
	\end{tabular}
	\caption{lena和elain图片的均值与方差}
	\label{mean_var}
\end{table}

\subsection{$lena$图像的插值}

我们分别对图像进行最邻近插值,双线性插值和双三次插值,分别得到了三幅不同的图像,我将从图像和算法两个方面对这三个结果进行展示,插值的结果如图所示\ref{interpolation_result}。

\begin{figure}[h!]
	\centering
	\subfigure[最邻近插值的图像]{
		\label{shear_elain1} %%first figure label
		\includegraphics[width = 0.3\textwidth]{elainnear.png}}
	\hspace{0.1in} \subfigure[双线性插值的图像]{
		\label{shear_elain2} %%second figure label
		\includegraphics[width = 0.3\textwidth]{elainlinear.png}}
	\hspace{0.1in} \subfigure[双三次插值的图像]{
		\label{shear_elain3} %%second figure label
		\includegraphics[width = 0.3\textwidth]{elaincubic.png}}
	\caption{插值结果} 
	\label{interpolation_result} %%label for entire figure
\end{figure}

我们可以对图片进行放大,观察其细节,如图\ref{interpolation_result2}所示。

\begin{figure}[h!]
	\centering
	\subfigure[最邻近插值]{
		\label{shear_elain1} %%first figure label
		\includegraphics[width = 0.3\textwidth]{near.png}}
	\hspace{0.1in} \subfigure[双线性插值]{
		\label{shear_elain2} %%second figure label
		\includegraphics[width = 0.3\textwidth]{linear.png}}
	\hspace{0.1in} \subfigure[双三次插值]{
		\label{shear_elain3} %%second figure label
		\includegraphics[width = 0.3\textwidth]{cubic.png}}
	\caption{插值结果放大} 
	\label{interpolation_result2} %%label for entire figure
\end{figure}

从图像细节中可以看出,最邻近插值的方法表现出的效果较差,区域模糊化明显,同时双线性内插的效果较最邻近内插有重大改进,而双三次内插的结果较双线性内插的结果稍微清晰一些,这和我们的设想一致。

\subsubsection{最邻近插值}

最邻近内插的方法是将原图像中最临近的灰度赋值给每个新的位置,这种方法简单,但是会有很多缺陷,如很多边缘的失真,由于这些缺陷,实际上这样的插值方法并不常用。

\subsubsection{双线性插值}

双线性内插的效果较最邻近内插有了明显的改观,其原理如下。

我们用$(x,y)$来表示我们想要赋以灰度值的位置,并令$v(x,y)$表示灰度,则对于双线性内插来说,赋值公式\ref{bilinear}为:

\begin{equation}\label{bilinear}
	v(x,y) = ax + by + cxy + d
\end{equation}

式中,4个系数可用由点$(x,y)$的四个最邻近的点写出的未知方程确定。不过,这也增加了插值的计算量\upcite{dip}。

\subsubsection{双三次插值}

双三次内插的效果较双线性内插有了更加清晰的效果,其赋值公式\ref{bicubic}为:

\begin{equation}\label{bicubic}
	v(x,y) = \sum_{i=0}^{3} \sum_{j=0}^{3} a_{i,j}x^{i}y^{j}
\end{equation}

式中,16个系数可用由点$(x,y)$的16个最邻近的点写出的未知方程确定。双三次插值在保持细节方面比双线性插值做的要更好\upcite{dip}。

\subsection{$lena$图像和$elain$图像的仿射变换和插值处理}

下面我们将介绍利用图像的仿射变换,对图像进行水平偏移错切和旋转的操作。最常用的空间坐标变化是仿射变换,其一般形式如公式\ref{Trans2}所示\upcite{dip}。
\begin{equation}\label{Trans2}
	\left( \begin{array}{ccc}
		x & y & 1
		\end{array} \right)
		=
	\left( \begin{array}{ccc}
		v & w & 1
		\end{array} \right) T
		=
	\left( \begin{array}{ccc}
		v & w & 1
		\end{array} \right) 
	\left( \begin{array}{ccc}
		t_{1,1}  & t_{1,2}  & 0  \\
		t_{2,1}  & t_{2,2} & 0 \\
		t_{3,1} & t_{3,2} & 1 
		\end{array} \right)
\end{equation}
式中,$(v,w)$是原图像中像素的坐标,$(x,y)$是变换后图像的坐标。通过仿射变换我们可以实现图像的水平偏移和旋转等操作。

\subsubsection{水平偏移}

水平偏移是通过图片的仿射变换来完成的,如公式\ref{trans}所示\upcite{dip}。
\begin{equation}\label{trans}
	\left( \begin{array}{ccc}
		x & y & 1
		\end{array} \right)
		=
	\left( \begin{array}{ccc}
		v & w & 1
		\end{array} \right) 
	\left( \begin{array}{ccc}
		1  & S_{h}  & 0  \\
		0  & 1 & 0 \\
		0 & 0 & 1 
		\end{array} \right)
\end{equation}

将$S_{h}$的值设为1.5,可以得到图\ref{shear_result1}和图\ref{shear_result2}。
\begin{figure}[h!]
	\centering
	\subfigure[水平shear后的图像]{
		\label{shear_lena}
		\includegraphics[width = 0.5\textwidth]{lenashearorigin.png}}\\
	\subfigure[最邻近插值后的图像]{
		\label{shear_lena1} %%first figure label
		\includegraphics[width = 0.25\textwidth]{lenashearnear.png}}
	\hspace{0.1in} \subfigure[双线性插值后的图像]{
		\label{shear_lena2} %%second figure label
		\includegraphics[width = 0.25\textwidth]{lenashearlinear.png}}
	\hspace{0.1in} \subfigure[双三次插值后的图像]{
		\label{shear_lena3} %%second figure label
		\includegraphics[width = 0.25\textwidth]{lenashearcubic.png}}
	\caption{lena图像水平偏移变换并插值的结果} 
	\label{shear_result1} %%label for entire figure
\end{figure}

\begin{figure}[h!]
	\centering
	\subfigure[水平shear后的图像]{
		\label{shear_elain}
		\includegraphics[width = 0.5\textwidth]{elainshearorigin.png}}\\
	\subfigure[最邻近插值后的图像]{
		\label{shear_elain1} %%first figure label
		\includegraphics[width = 0.25\textwidth]{elainshearnear.png}}
	\hspace{0.1in} \subfigure[双线性插值后的图像]{
		\label{shear_elain2} %%second figure label
		\includegraphics[width = 0.25\textwidth]{elainshearlinear.png}}
	\hspace{0.1in} \subfigure[双三次插值后的图像]{
		\label{shear_elain3} %%second figure label
		\includegraphics[width = 0.25\textwidth]{elainshearcubic.png}}
	\caption{elain图像水平偏移变换并插值的结果} 
	\label{shear_result2} %%label for entire figure
\end{figure}

\subsubsection{旋转}

旋转变换也是通过图片的仿射变换来完成的,如公式\ref{trans2}所示\upcite{dip}。
\begin{equation}\label{trans2}
	\left( \begin{array}{ccc}
		x & y & 1
		\end{array} \right)
		=
	\left( \begin{array}{ccc}
		v & w & 1
		\end{array} \right) 
	\left( \begin{array}{ccc}
		cos\theta  & sin\theta  & 0  \\
		-sin\theta  & cos\theta & 0 \\
		0 & 0 & 1 
		\end{array} \right)
\end{equation}
将$\theta$的值设为30度,可以得到图\ref{rotation_result1}和图\ref{rotation_result2}。

\begin{figure}[h!]
	\centering
	\subfigure[最邻近插值后的图像]{
		\label{rotate_lena1} %%first figure label
		\includegraphics[width = 0.25\textwidth]{lenarotationnear.png}}
	\hspace{0.1in} \subfigure[双线性插值后的图像]{
		\label{rotate_lena2} %%second figure label
		\includegraphics[width = 0.25\textwidth]{lenarotationlinear.png}}
	\hspace{0.1in} \subfigure[双三次插值后的图像]{
		\label{rotate_lena3} %%second figure label
		\includegraphics[width = 0.25\textwidth]{lenarotationcubic.png}}
	\caption{lena图像旋转30度变换并插值的结果} 
	\label{rotation_result1} %%label for entire figure
\end{figure}

\begin{figure}[h!]
	\centering
	\subfigure[最邻近插值后的图像]{
		\label{rotate_elain1} %%first figure label
		\includegraphics[width = 0.25\textwidth]{elainrotationnear.png}}
	\hspace{0.1in} \subfigure[双线性插值后的图像]{
		\label{rotate_elain2} %%second figure label
		\includegraphics[width = 0.25\textwidth]{elainrotationlinear.png}}
	\hspace{0.1in} \subfigure[双三次插值后的图像]{
		\label{rotate_elain3} %%second figure label
		\includegraphics[width = 0.25\textwidth]{elainotationcubic.png}}
	\caption{elain图像旋转30度变换并插值的结果} 
	\label{rotation_result2} %%label for entire figure
\end{figure}

\section{小结}

本次作业,通过实践,我深入了解了图像处理的相关基础知识,同时也提高了动手能力,加深了理解和掌握程度,也让我拥有了解决实际问题的基本功。同时,代码能力也得到了有效锻炼,
我的python能力得到了有效锻炼,同时也学习了opencv的很多基本功能。我自行组织的$cv toolbox$在我的$github$上维护,欢迎访问 \newline $https://github.com/1989Ryan/Digital-Image-Processing-Project$进行了解.

\newpage

\begin{thebibliography}{}
    \bibitem{MSdoc} Microsoft Windows Dev Center, 
    Windows GDI, ``Bitmap", [online] available: 
    \newline $https://docs.microsoft.com/en-us/windows/desktop/gdi/bitmaps$  
    (Febrary, 26, 2019).

	\bibitem{Wiki} Wikipedia, BMP file format, [online] available: \newline 
	$https://en.wikipedia.org/wiki/BMP\_file\_format$ (Febrary, 26, 2019).

	\bibitem{dip} Rafael C. Gonzalez,Richard E. Woods,数字图像处理. 北京: 电子工业出版社, 2017.5.

\end{thebibliography}

\clearpage

\end{document}